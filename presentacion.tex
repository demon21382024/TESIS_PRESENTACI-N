\documentclass[10pt,aspectratio=169]{beamer}

% ==============================
%     TEMAS Y PAQUETES
% ==============================

\usetheme{Madrid}
\usecolortheme{default}

\usepackage{graphicx}
\usepackage{amsmath}
\usepackage{hyperref}
\usepackage[utf8]{inputenc}
\usepackage[T1]{fontenc}

% Quitar barra inferior de navegación
\setbeamertemplate{navigation symbols}{}

% ==============================
%     DATOS DE LA PRESENTACIÓN
% ==============================

\title[Re-ID Autosupervisado]{\textbf{Reidentificación de Personas Aplicando Aprendizaje Autosupervisado}}
\author[H. Canto, J. Torres]{Harold Canto \\ Juan Torres \\[0.2cm]
\small Asesor: Edward Jorge Yuri Cayllahua Cahuina}
\institute[UTEC]{Universidad de Ingeniería y Tecnología (UTEC)}
\date{2025}

% ==============================
%             DOCUMENTO
% ==============================

\begin{document}

% TÍTULO
\begin{frame}
    \titlepage
\end{frame}

% ==============================
%        SLIDE 1: TEMA
% ==============================

\begin{frame}{Tema}
Reidentificación de personas aplicando aprendizaje autosupervisado.
\end{frame}

% ==============================
%        SLIDE 2: ÍNDICE
% ==============================

\begin{frame}{Índice}
\begin{enumerate}
    \item Motivación
    \item Problema Computacional
    \item Justificación
    \item Objetivo General y Específicos
    \item Referencias
\end{enumerate}
\end{frame}

% ==============================
%        SLIDE 3: MOTIVACIÓN
% ==============================

\begin{frame}{Motivación}
\begin{itemize}
    \item Person Re-ID identifica personas en distintas cámaras sin depender del rostro.
    \item Métodos supervisados: alto rendimiento, pero requieren mucho etiquetado manual.
    \item PersonViT: robusto contra oclusiones (MIM + DINO + ViT), pero computacionalmente costoso.
    \item ISR: aprendizaje autosupervisado con pares inter-frame y contraste (sin etiquetas).
    \item El autosupervisado permite reducir costos de anotación y mejorar generalización.
\end{itemize}

\footnotesize Fuente: Person Re-Identification - an overview | ScienceDirect Topics
\end{frame}

% ==============================
%    SLIDE 4: PROBLEMA COMPUTACIONAL
% ==============================

\begin{frame}{Problema Computacional}
¿Cómo lograr que el aprendizaje autosupervisado aprenda
representaciones de identidad en datos no etiquetados que funcionen
entre distintas cámaras y condiciones (iluminación, pose, oclusión),
logrando un rendimiento comparable al supervisado y con menor costo
computacional?
\end{frame}

% ==============================
%         SLIDE 5: JUSTIFICACIÓN
% ==============================

\begin{frame}{Justificación}

\textbf{Técnica:}
\begin{itemize}
    \item El aprendizaje supervisado requiere grandes datasets etiquetados.
    \item Los métodos autosupervisados (ISR, PersonViT) han mostrado
    resultados competitivos.
\end{itemize}

\vspace{0.3cm}

\textbf{Práctica:}
\begin{itemize}
    \item Reduce costos operativos.
    \item Es escalable para sistemas de vigilancia real.
    \item Útil en escenarios urbanos con múltiples cámaras.
\end{itemize}
\end{frame}

% ==============================
%     SLIDE 6: OBJETIVO GENERAL
% ==============================

\begin{frame}{Objetivo General}
Proponer un método autosupervisado para reidentificación de personas,
evaluando su capacidad de reducir la dependencia de datos etiquetados y
mejorar la generalización en condiciones reales.
\end{frame}

% ==============================
%   SLIDE 7: OBJETIVOS ESPECÍFICOS
% ==============================

\begin{frame}{Objetivos Específicos}
\begin{itemize}
    \item Seleccionar datasets adecuados para entrenamiento y evaluación.
    \item Definir métricas relevantes: Rank-1, mAP.
    \item Explorar técnicas autosupervisadas aplicables a Re-ID.
    \item Comparar el enfoque autosupervisado con métodos supervisados.
    \item Evaluar escalabilidad y desafíos prácticos.
\end{itemize}
\end{frame}

% ==============================
%         SLIDE 8: REFERENCIAS
% ==============================

\begin{frame}{Referencias}
\footnotesize
\begin{itemize}
    \item C. Joshua et al., “Using optical flow consistency for self-supervised person Re-ID”, 2025.
    \item X. Liu et al., “UCM-VeID V2: Multi-View End-to-End Re-ID”, CVPR 2025.
    \item M. Varenyk et al., “Self-supervised low-FPS tracking for Re-ID”, 2025.
    \item H. Rao et al., “Self-supervised gait encoding for Re-ID”, IJCAI 2020.
    \item Z. Dou et al., “Identity-seeking self-supervised Re-ID”, ICCV 2023.
\end{itemize}
\end{frame}

% ==============================
%              FINAL
% ==============================

\begin{frame}
    \centering\Huge Gracias
\end{frame}

\end{document}
