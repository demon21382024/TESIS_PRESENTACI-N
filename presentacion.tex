\documentclass[12pt,aspectratio=169]{beamer}

% ==============================
%    TEMAS Y PAQUETES
% ==============================

\usetheme{Madrid}
\usecolortheme{default}

\usepackage{graphicx}
\usepackage{amsmath}
\usepackage{hyperref}
\usepackage[utf8]{inputenc}
\usepackage[T1]{fontenc}
\usepackage{comment}

% Quitar barra inferior de navegación
\setbeamertemplate{navigation symbols}{}

% ==============================
%    DATOS DE LA PRESENTACIÓN
% ==============================

\title[Re-ID Autosupervisado]{\textbf{Reidentificación de Personas basado en reconocimiento de marcha}}
\author[H. Canto, J. Torres]{Harold Canto \\ Juan Torres \\[0.2cm]
\small Asesor: Edward Jorge Yuri Cayllahua Cahuina}
\institute[UTEC]{Universidad de Ingeniería y Tecnología (UTEC)}
\date{2025}

% ==============================
%            DOCUMENTO
% ==============================

\begin{document}

% TÍTULO
\begin{frame}
    \titlepage
\end{frame}

% ==============================
%    SLIDE CONTENIDO (ÍNDICE)
% ==============================

\begin{frame}{Contenido}
    \tableofcontents
\end{frame}

% ==============================
%           SECCION 1
% ==============================

\section{Motivación}

\begin{frame}{Motivación}
\begin{itemize}
  \item Person Re-ID busca identificación entre cámaras sin depender solo del rostro.
  \item Los métodos supervisados alcanzan alto rendimiento pero requieren costosos conjuntos etiquetados.
  \item Métodos autosupervisados basados en consistencia temporal o pares frame/tracklet pueden aprender invariancias útiles sin etiquetas.
  \item Un enfoque híbrido puede aprovechar lo mejor de ambos mundos.
\end{itemize}
\footnotesize Fuente: Person Re-Identification - an overview | ScienceDirect Topics
\end{frame}

% ==============================
%           SECCION 2
% ==============================

\section{Problema Computacional}

\begin{frame}{Problema Computacional}
¿Cómo lograr que el aprendizaje autosupervisado capture representaciones de identidad en secuencias de marcha no etiquetadas, que sean robustas, y que al combinarse con ajuste fino supervisado alcancen rendimiento competitivo con menores costos de etiquetado y recursos?
\end{frame}

% ==============================
%           SECCION 3
% ==============================

\section{Justificación}

\begin{frame}{Justificación}

\textbf{Técnica:}
\begin{itemize}
    \item El supervisado necesita etiquetas y el SSL aprovecha grandes volúmenes no etiquetados.
  \item Un preentrenamiento SSL adecuado puede mejorar la generalización entre cámaras.
\end{itemize}

\vspace{0.3cm}

\textbf{Práctica:}
\begin{itemize}
    \item Reduce costos operativos.
    \item Escalable para vigilancia real en entornos urbanos.
    \item Mejora robustez en condiciones de baja resolución.
\end{itemize}

\end{frame}

% ==============================
%           SECCION 4
% ==============================

\section{Objetivos}

\begin{frame}{Objetivo General}
Desarrollar y evaluar un método híbrido para Re-ID basado en gait, que reduzca la dependencia de datos etiquetados y mejore la generalización en condiciones reales.
\end{frame}

\begin{frame}{Objetivos Específicos}
\begin{itemize}
  \item Seleccionar datasets y protocolo.
  \item Diseñar un pretext-task SSL orientado a secuencias de marcha.
  \item Implementar fine-tuning con pérdidas métricas (Triplet) y clasificación (CrossEntropy).
  \item Evaluar con métricas según estado del arte.
\end{itemize}
\end{frame}

% ==============================
%    SLIDE: PIPELINE
% ==============================

\section{Pipeline Metodológico}

\begin{frame}{Pipeline Metodológico}
\centering
% Asegúrate de tener la imagen pipeline.jpeg en la carpeta
% Si no la tienes, comenta la siguiente línea poniendo un % al inicio
%\includegraphics[width=0.42\textwidth]{pipeline.jpeg}
%\vspace{0.3cm}
%{\footnotesize \textbf{Fuente:} Elaboración propia.}

\begin{itemize}
  \item Fase I: Preentrenamiento SSL (pares temporales / augmentaciones).
  \item Fase II: Transferencia y fine-tuning supervisado (PK Sampler, Triplet, CrossEntropy).
  \item Evaluación bajo Rank-1 y mAP, junto con pruebas de escalabilidad por data.
\end{itemize}

\end{frame}

\begin{frame}{Motivación del Modelo Híbrido}
\begin{itemize}
    \item El aprendizaje autosupervisado captura patrones robustos sin necesidad de etiquetas.
    \item El aprendizaje supervisado refina las representaciones para la tarea específica de Re-ID.
    \item El modelo híbrido combina ambos enfoques para mejorar generalización y rendimiento.
    \item Esta estrategia reduciría la dependencia de datos etiquetados sin sacrificar exactitud.
\end{itemize}
\end{frame}

% ==============================
%           SECCION 5
% ==============================

\section{Dataset y Preprocesamiento}

\begin{frame}{Dataset}
\textbf{CASIA-B} ofrece una secuencia completa de imágenes para el estudio de \textit{gait recognition}.
\begin{itemize}
    \item 124 sujetos, 3 condiciones (normal, bolso y abrigo).
    \item Capturas desde 11 vistas angulares entre 0° y 180°.
    \item Adecuado para secuencias de marcha y SSL temporal.
\end{itemize}
\end{frame}

\begin{frame}{Preprocesamiento}
\begin{itemize}
    \item Conversión a escala de grises; normalización de píxeles a [0,1].
    \item Estandarización a size de 64×64.
    \item Augmentaciones para SSL: recorte, rotación leve, flip horizontal, cambios leves de brillo/contraste.
\end{itemize}
\end{frame}

\begin{frame}{Arquitectura General del Modelo Híbrido}
\textbf{Etapas del híbrido:}
\begin{enumerate}
  \item Pre-entrenamiento SSL: backbone CNN aprende representaciones temporales.
  \item Transferencia: pesos de SSL como inicialización del backbone.
  \item Fine-tuning supervisado: cabeza de clasificación + cabeza métrica (normalizada), optimizadas con CrossEntropy + Triplet Loss.
\end{enumerate}
\end{frame}

\begin{frame}{Componente autosupervisado (SSL)}
\begin{itemize}
    \item Basado en contraste temporal entre frames de la misma secuencia.
    \item Utiliza proyección SimCLR y pérdida InfoNCE para pares temporales.
    \item Augmentaciones aplicadas solo durante preentrenamiento SSL.
\end{itemize}
\end{frame}

\begin{frame}{Componente supervisado}
\begin{itemize}
    \item Se añade una cabeza de clasificación sobre el embedding del SSL.
    \item Se usa Triplet Loss para maximizar separación entre identidades.
    \item Se emplea \textbf{P-K Sampler} para mejorar el aprendizaje métrico.
    \item Ajuste fino del backbone para adaptar las características al problema Re-ID.
\end{itemize}
\end{frame}

\begin{frame}{Módulo de integración híbrida}
\textbf{El modelo híbrido combina:}
\begin{itemize}
    \item Representaciones autosupervisadas (\textit{global invariances}).
    \item Ajuste supervisado para la identidad (\textit{local discriminative features}).
\end{itemize}

\textbf{Algunas ventajas del modelo híbrido:}
\begin{itemize}
    \item Mejora Rank-1 y mAP sin necesidad de más datos.
    \item Obtiene embeddings más consistentes entre cámaras distintas.
\end{itemize}
\end{frame}

\section{Resultados}

\begin{frame}{Resultados Cuantitativos}
\centering
\begin{tabular}{lccc}
\hline
Modelo & Rank-1 & mAP \\
\hline
Supervisado desde cero & 62.1 & 55.3 \\
SSL + Supervisado (propuesto) & 91.29 & 44.83 \\
\hline
\end{tabular}
\end{frame}

% ==============================
%           SECCION 6
% ==============================
\section{Conclusiones y Recomendaciones}

\begin{frame}{Conclusiones}
    \begin{itemize}
        \setlength\itemsep{1em}
        \item El modelo híbrido alcanzó una precisión Rank-1 del 91.29\%, validando que el preentrenamiento autosupervisado reduce eficazmente la dependencia de grandes volúmenes de datos etiquetados.
        \item La discrepancia observada en el mAP (44.83\%) se atribuye a la pérdida de información dinámica en ángulos de visión extremos ($0^\circ$ y $180^\circ$), donde la silueta pierde discriminabilidad.
        \item El sistema demostró estabilidad algorítmica al escalar la población de prueba, manteniendo el rendimiento en el rango del 88-91\% a pesar del incremento de distractores.
        \item Se confirmó que la fase autosupervisada actúa como un catalizador eficiente, aunque la inicialización de la capa de clasificación requiere estabilización para evitar óptimos locales.
    \end{itemize}
\end{frame}

\section{Trabajo Futuro}

\begin{frame}{Trabajo Futuro y Recomendaciones}
    \begin{itemize}
        \setlength\itemsep{1em}
        \item Se recomienda implementar \textit{Stochastic Weight Averaging} (SWA) para mitigar la sensibilidad a la inicialización y reducir la varianza durante el ajuste fino.
        \item Es necesario integrar mecanismos de atención espacial (View-Aware) para ponderar dinámicamente las regiones visibles y mejorar la recuperación en vistas frontales.
        \item Se sugiere transitar hacia arquitecturas \textit{Vision Transformers} (ViT) para capturar dependencias espacio-temporales globales superiores a las CNNs tradicionales.
        \item Se propone investigar la fusión multimodal con datos RGB o esqueletos para compensar la falta de información de la silueta en condiciones de oclusión severa.
    \end{itemize}
\end{frame}

% ==============================
%           FINAL
% ==============================

\section{Referencias}

\begin{frame}{Referencias}
\footnotesize
\begin{itemize}
    \item C. Joshua et al., “Using optical flow consistency for self-supervised person Re-ID”, 2025.
    \item X. Liu et al., “UCM-VeID V2: Multi-View End-to-End Re-ID”, CVPR 2025.
    \item M. Varenyk et al., “Self-supervised low-FPS tracking for Re-ID”, 2025.
    \item H. Rao et al., “Self-supervised gait encoding for Re-ID”, IJCAI 2020.
    \item Z. Dou et al., “Identity-seeking self-supervised Re-ID”, ICCV 2023.
\end{itemize}
\end{frame}

% ==============================
%              FINAL
% ==============================

\begin{frame}
    \centering\Huge Gracias
\end{frame}

\end{document}