\documentclass[12pt,aspectratio=169]{beamer}

% ==============================
%     TEMAS Y PAQUETES
% ==============================

\usetheme{Madrid}
\usecolortheme{default}

\usepackage{graphicx}
\usepackage{amsmath}
\usepackage{hyperref}
\usepackage[utf8]{inputenc}
\usepackage[T1]{fontenc}
\usepackage{comment}

% Quitar barra inferior de navegación
\setbeamertemplate{navigation symbols}{}

% ==============================
%     DATOS DE LA PRESENTACIÓN
% ==============================

\title[Re-ID Autosupervisado]{\textbf{Reidentificación de Personas Aplicando Aprendizaje Autosupervisado}}
\author[H. Canto, J. Torres]{Harold Canto \\ Juan Torres \\[0.2cm]
\small Asesor: Edward Jorge Yuri Cayllahua Cahuina}
\institute[UTEC]{Universidad de Ingeniería y Tecnología (UTEC)}
\date{2025}

% ==============================
%             DOCUMENTO
% ==============================

\begin{document}

% TÍTULO
\begin{frame}
    \titlepage
\end{frame}

% ==============================
%     SLIDE CONTENIDO (ÍNDICE)
% ==============================

\begin{frame}{Contenido}
    \tableofcontents
\end{frame}

% ==============================
%           SECCION 1
% ==============================

\section{Motivación}

\begin{frame}{Motivación}
\begin{itemize}
    \item Person Re-ID identifica personas entre cámaras sin depender del rostro.
    \item Supervisado: excelente rendimiento, alto costo de etiquetado.
    \item PersonViT: robusto a oclusiones pero muy costoso computacionalmente.
    \item ISR: autosupervisado con contraste inter-frames (sin etiquetas).
    \item Autosupervisado reduce costos y mejora generalización.
\end{itemize}

\footnotesize Fuente: Person Re-Identification - an overview | ScienceDirect Topics
\end{frame}

% ==============================
%           SECCION 2
% ==============================

\section{Problema Computacional}

\begin{frame}{Problema Computacional}
¿Cómo lograr que el aprendizaje autosupervisado aprenda representaciones
de identidad en datos no etiquetados que funcionen en distintas cámaras,
condiciones de iluminación, pose y oclusión, alcanzando rendimiento 
cercano al supervisado pero con menor costo computacional?
\end{frame}

% ==============================
%           SECCION 3
% ==============================

\section{Justificación}

\begin{frame}{Justificación}

\textbf{Técnica:}
\begin{itemize}
    \item El supervisado requiere grandes datasets etiquetados.
    \item Técnicas autosupervisadas han demostrado resultados competitivos.
\end{itemize}

\vspace{0.3cm}

\textbf{Práctica:}
\begin{itemize}
    \item Reduce costos operativos.
    \item Escalable para vigilancia real en entornos urbanos.
    \item Menor dependencia de anotaciones manuales.
\end{itemize}

\end{frame}

% ==============================
%           SECCION 4
% ==============================

\section{Objetivos}

\begin{frame}{Objetivo General}
Proponer un método autosupervisado para reidentificación de personas,
evaluando su capacidad de reducir dependencia de datos etiquetados y
mejorar la generalización en condiciones reales.
\end{frame}

\begin{frame}{Objetivos Específicos}
\begin{itemize}
    \item Seleccionar datasets adecuados para entrenamiento y evaluación.
    \item Explorar técnicas autosupervisadas aplicables a Re-ID.
    \item Comparar enfoque autosupervisado vs supervisado.
    \item Definir métricas relevantes: Rank-1, mAP.
    \item Evaluar escalabilidad y desafíos prácticos.
\end{itemize}
\end{frame}

% ==============================
%     SLIDE: PIPELINE
% ==============================

\section{Pipeline Metodológico}

\begin{frame}{Pipeline Metodológico}
\centering
\includegraphics[width=0.42\textwidth]{pipeline.jpeg}

\vspace{0.3cm}
{\footnotesize \textbf{Fuente:} Elaboración propia.}
\end{frame}

\begin{frame}{Motivación del Modelo Híbrido}
\begin{itemize}
    \item El aprendizaje autosupervisado captura patrones robustos sin necesidad de etiquetas.
    \item El aprendizaje supervisado refina las representaciones para la tarea específica de Re-ID.
    \item El modelo híbrido combina ambos enfoques para mejorar generalización y rendimiento.
    \item Esta estrategia reduce la dependencia de datos etiquetados sin sacrificar exactitud.
\end{itemize}
\end{frame}

% ==============================
%           SECCION 5
% ==============================

\section{Dataset y Preprocesamiento}

\begin{frame}{Dataset}
\textbf{CASIA-B} ofrece una secuencia completa de imágenes para el estudio de \textit{gait recognition}.
\begin{itemize}
    \item 124 sujetos, 3 condiciones (normal, bolso, abrigo).
    \item 11 cámaras desde distintos ángulos.
    \item Secuencias de marcha ideales para SSL temporal.
\end{itemize}
\end{frame}

\begin{comment}
\begin{frame}{Preprocesamiento}
\begin{itemize}
    \item Conversión a escala de grises.
    \item Normalización a tamaño 64×64.
    \item Extracción de siluetas mediante segmentación.
    \item División por secuencias para contrastive learning.
\end{itemize}
\end{frame}
\end{comment}

\begin{frame}{Arquitectura General del Modelo Híbrido}
\textbf{Etapas del híbrido:}
\begin{enumerate}
    \item Preentrenamiento SSL: el backbone aprende representaciones invariantes.
    \item Transferencia: se reutilizan pesos aprendidos como inicialización.
    \item Fine-tuning Supervisado: se optimiza con Triplet Loss + softmax.
\end{enumerate}
\end{frame}

\begin{frame}{Componente autosupervisado (SSL)}
\begin{itemize}
    \item Basado en contraste temporal entre frames de la misma secuencia.
    \item Utiliza proyección SimCLR y pérdida InfoNCE.
    \item Aprende representaciones invariantes a iluminación y perspectiva.
    \item Actúa como base robusta para la etapa supervisada.
\end{itemize}
\end{frame}

\begin{frame}{Componente supervisado}
\begin{itemize}
    \item Se añade una cabeza de clasificación sobre el embedding del SSL.
    \item Se usa Triplet Loss para maximizar separación entre identidades.
    \item Se emplea \textbf{P-K Sampler} para mejorar el aprendizaje métrico.
    \item Ajuste fino del backbone para adaptar las características al problema Re-ID.
\end{itemize}
\end{frame}

\begin{frame}{Módulo de integración híbrida}
\textbf{El modelo híbrido combina:}
\begin{itemize}
    \item Representaciones autosupervisadas (\textit{global invariances}).
    \item Ajuste supervisado para la identidad (\textit{local discriminative features}).
\end{itemize}

\textbf{Algunas ventajas del modelo híbrido:}
\begin{itemize}
    \item Mejora Rank-1 y mAP sin necesidad de más datos.
    \item Obtiene embeddings más consistentes entre cámaras distintas.
\end{itemize}
\end{frame}

\section{Resultados}

\begin{frame}{Resultados Cuantitativos}
\centering
\begin{tabular}{lccc}
\hline
Modelo & Rank-1 & mAP \\
\hline
Supervisado desde cero & 62.1 & 55.3 \\
SSL + Supervisado (propuesto) & 74.8 & 68.4 \\
\hline
\end{tabular}
\end{frame}

\section{Conclusiones}

\begin{frame}{Conclusiones}
\begin{itemize}
    \item El aprendizaje autosupervisado permite aprender identidad sin etiquetas.
    \item El modelo SSL + supervisado supera al modelo únicamente supervisado.
    \item La arquitectura propuesta es ligera y escalable para entornos reales.
\end{itemize}
\end{frame}

\begin{frame}{Trabajo futuro}
\begin{itemize}
    \item Extender el método a escenas RGB y a cámaras de vigilancia reales.
    \item Explorar arquitecturas basadas en Transformers.
    \item Automatizar pseudo-etiquetado con clustering dinámico.
    \item Evaluar generalización con benchmarks estándar.
\end{itemize}
\end{frame}

% ==============================
%           FINAL
% ==============================

\section{Referencias}

\begin{frame}{Referencias}
\footnotesize
\begin{itemize}
    \item C. Joshua et al., “Using optical flow consistency for self-supervised person Re-ID”, 2025.
    \item X. Liu et al., “UCM-VeID V2: Multi-View End-to-End Re-ID”, CVPR 2025.
    \item M. Varenyk et al., “Self-supervised low-FPS tracking for Re-ID”, 2025.
    \item H. Rao et al., “Self-supervised gait encoding for Re-ID”, IJCAI 2020.
    \item Z. Dou et al., “Identity-seeking self-supervised Re-ID”, ICCV 2023.
\end{itemize}
\end{frame}

% ==============================
%              FINAL
% ==============================

\begin{frame}
    \centering\Huge Gracias
\end{frame}

\end{document}
