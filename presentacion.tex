\documentclass[12pt,aspectratio=169]{beamer}

% ==============================
%     TEMAS Y PAQUETES
% ==============================

\usetheme{Madrid}
\usecolortheme{default}

\usepackage{graphicx}
\usepackage{amsmath}
\usepackage{hyperref}
\usepackage[utf8]{inputenc}
\usepackage[T1]{fontenc}

% Quitar barra inferior de navegación
\setbeamertemplate{navigation symbols}{}

% ==============================
%     DATOS DE LA PRESENTACIÓN
% ==============================

\title[Re-ID Autosupervisado]{\textbf{Reidentificación de Personas Aplicando Aprendizaje Autosupervisado}}
\author[H. Canto, J. Torres]{Harold Canto \\ Juan Torres \\[0.2cm]
\small Asesor: Edward Jorge Yuri Cayllahua Cahuina}
\institute[UTEC]{Universidad de Ingeniería y Tecnología (UTEC)}
\date{2025}

% ==============================
%             DOCUMENTO
% ==============================

\begin{document}

% TÍTULO
\begin{frame}
    \titlepage
\end{frame}

% ==============================
%     SLIDE CONTENIDO (ÍNDICE)
% ==============================

\begin{frame}{Contenido}
    \tableofcontents
\end{frame}

% ==============================
%           SECCION 1
% ==============================

\section{Motivación}

\begin{frame}{Motivación}
\begin{itemize}
    \item Person Re-ID identifica personas entre cámaras sin depender del rostro.
    \item Supervisado: excelente rendimiento, alto costo de etiquetado.
    \item PersonViT: robusto a oclusiones pero muy costoso computacionalmente.
    \item ISR: autosupervisado con contraste inter-frames (sin etiquetas).
    \item Autosupervisado reduce costos y mejora generalización.
\end{itemize}

\footnotesize Fuente: Person Re-Identification - an overview | ScienceDirect Topics
\end{frame}

% ==============================
%           SECCION 2
% ==============================

\section{Problema Computacional}

\begin{frame}{Problema Computacional}
¿Cómo lograr que el aprendizaje autosupervisado aprenda representaciones
de identidad en datos no etiquetados que funcionen en distintas cámaras,
condiciones de iluminación, pose y oclusión, alcanzando rendimiento 
cercano al supervisado pero con menor costo computacional?
\end{frame}

% ==============================
%           SECCION 3
% ==============================

\section{Justificación}

\begin{frame}{Justificación}

\textbf{Técnica:}
\begin{itemize}
    \item El supervisado requiere grandes datasets etiquetados.
    \item Técnicas autosupervisadas han demostrado resultados competitivos.
\end{itemize}

\vspace{0.3cm}

\textbf{Práctica:}
\begin{itemize}
    \item Reduce costos operativos.
    \item Escalable para vigilancia real en entornos urbanos.
    \item Menor dependencia de anotaciones manuales.
\end{itemize}

\end{frame}

% ==============================
%           SECCION 4
% ==============================

\section{Objetivos}

\begin{frame}{Objetivo General}
Proponer un método autosupervisado para reidentificación de personas,
evaluando su capacidad de reducir dependencia de datos etiquetados y
mejorar la generalización en condiciones reales.
\end{frame}

\begin{frame}{Objetivos Específicos}
\begin{itemize}
    \item Seleccionar datasets adecuados para entrenamiento y evaluación.
    \item Definir métricas relevantes: Rank-1, mAP.
    \item Explorar técnicas autosupervisadas aplicables a Re-ID.
    \item Comparar enfoque autosupervisado vs supervisado.
    \item Evaluar escalabilidad y desafíos prácticos.
\end{itemize}
\end{frame}

% ==============================
%     SLIDE: PIPELINE
% ==============================

\section{Pipeline Metodológico}

\begin{frame}{Pipeline Metodológico}
\centering
\includegraphics[width=0.42\textwidth]{pipeline.jpeg}

\vspace{0.3cm}
{\footnotesize \textbf{Fuente:} Elaboración propia.}
\end{frame}

% ==============================
%           SECCION 5
% ==============================

\section{Referencias}

\begin{frame}{Referencias}
\footnotesize
\begin{itemize}
    \item C. Joshua et al., “Using optical flow consistency for self-supervised person Re-ID”, 2025.
    \item X. Liu et al., “UCM-VeID V2: Multi-View End-to-End Re-ID”, CVPR 2025.
    \item M. Varenyk et al., “Self-supervised low-FPS tracking for Re-ID”, 2025.
    \item H. Rao et al., “Self-supervised gait encoding for Re-ID”, IJCAI 2020.
    \item Z. Dou et al., “Identity-seeking self-supervised Re-ID”, ICCV 2023.
\end{itemize}
\end{frame}

% ==============================
%              FINAL
% ==============================

\begin{frame}
    \centering\Huge Gracias por su atención.
\end{frame}

\end{document}
